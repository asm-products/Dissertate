 ოპტიმიზირებულია სამეცნიერო-კვლევითი ინსტიტუტების ქსელი  – კვლევითი ინსტიტუტები ინტეგრირებულნი არიან უნივერსიტეტებში;
-	შექმნილია 5 სამეცნიერო-კვლევითი ცენტრი;
-	დაფუძნებულია 6 სახელმწიფო სამეცნიერო-კვლევითი ინსტიტუტი;
-	სამეცნიერო საქმიანობის შეფასება მოხდება საერთაშორისო დონეზე;
-	იზრდება სამეცნიერო საქმიანობის დაფინანსება.
სამეცნიერო-კვლევითი საქმიანობის გაუმჯობესება მოიცავს უმაღლესი განათლების მესამე საფეხურის დოქტორანტურის განვითარების ხელშეწყობას, მისი დაფინანსების უზრუნველყოფას. სახელმწიფო აფინანსებს იმ დოქტორანტებს, რომელთაც დაიკავეს „საბიუჯეტო“ (დაფინანსებული) ადგილები კონკურსის გავლის შემდეგ. საბიუჯეტო ადგილებს კი ანაწილებს მეცნიერებისა და განათლების სამინისტრო დოქტორანტურის ინსტიტუციებს  შორის. ის პირები, რომლებიც კონკურსის საფუძველზე ვერ მოხვდებიან სახელმწიფოს მიერ დაფინანსებულ ადგილებზე, დოქტორატურის კომიტეტის  შეთავაზებით, შუძლიათ აირჩიონ ფასიანი სწავლება დოქტორანტის ინსტიტუციაში, მასთან ხელშკრულების დადების საფუძველზე.  
სახელმწოფო დაფინანსებიდან მიღებული თანხები ნაწილდება დოქტორანტურის განხორციელებაზე პასუხისმგებელ ინსტიტუციებს შორის ცალკეული ხელშეკრულებების საფუძველზე. დოქტორანტურა ფინანსდება სახელმწიფო ბიუჯეტიდან გამოყოფილი ასიგნებებიდან, სახელმწიფო საინვესტიციო პროგრამების და პროექტების თანხებიდან, სწავლების გადასახადების შემოსავლებიდან, სამეცნიერო, ბიზნეს-საქმინობის და გაწეული მომსახურების შემოსავლებიდან, სამეცნიერო კვლევების საკონკურსო პროგრამული დაფინანსების თანხებიდან, სახელმწიფო, საერთაშორისო და უცხოური ფონდების და ორგანიზაციების მიერ გამოყოფილი თანხებიდან და სხვა კანონიერი გზით მოპოვებული თანხებიდან.
დოქტორანტის ინსტიტუცია სახელმწიფოს მიერ დაფინანსებულ ადგილებზე ჩარიცხულ დოქტორანტებს სწავლების დროს უნიშნავს სტიპენდიებს, რომელთა ოდენობას ადგენს ლიტვის მთავრობა. სტიპენდიას გასცემს დოქტორანტის 
